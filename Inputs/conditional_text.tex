\section{Conditional Text}

It can be helpful to have text in the source document rendered only when certain conditions are true. A common example is when a generated document is intended for multiple audiences, but not all text is of interest to each audience. So selective rendering of the text in the generated document is desirable.

To employ conditional text, the {\itshape ifthen} package can be loaded. This package enables these commands to be used:

\begin{verbatim}
   \ifthenelse{<test>}{<then_block>}{<else_block>}
   \whiledo{<test>}{<do_block>}
\end{verbatim}

These behave similarly to if-then-else conditional blocks and while loops in programming languages. For either command, there is a test condition. For \verb2\ifthenelse2, the \linebreak\ $<$then\_block$>$ is executed if the test is true and the $<$else\_block$>$ otherwise. For \linebreak \verb2\whiledo2, the $<$do\_block$>$ is executed so long as the test is true. Within the body of the while script, one must alter a value that is tested so the loop will end.

The Boolean commands are useful for setting up conditions to test:

\begin{verbatim}
   \newboolean{<name>}
   \setboolean{<name>}{<value>}
   \boolean{<name>}
\end{verbatim}

The first two commands initialize a Boolean flag and give it a truth value. The final command returns the value of a Boolean flag.

The {\itshape equal} command can be used in conjunction with a Boolean flag:

\ind\verb3\equal{<string1>}{<string2>}3 

Here is an example:

\begin{verbatim}
   \newboolean{cond1}
   \setboolean{cond1}{true}
   \ifthenelse{\equal{\boolean{cond1}}{true}}
     {This is printed.}
     {This isn't.}
\end{verbatim}
